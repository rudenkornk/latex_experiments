\documentclass[a4paper,12pt]{article}
% 13pt size font
\usepackage[fontsize=12pt]{scrextend}

%=============================== Packages ===============================

% Encodings
\usepackage[T1,T2A]{fontenc}
\usepackage{fontspec}
\usepackage[english,russian]{babel}


% NewDocumentCommand and conditional expressions
\usepackage{xparse}
\usepackage{iftex}

\usepackage[usenames,dvipsnames,svgnames,table,rgb]{xcolor}

\usepackage{enumitem}

\usepackage{multicol}

% Margins
\usepackage{geometry}

% Centered sections
\usepackage{sectsty}

% More intelligent hyphenation
\usepackage{microtype}

% Tables
\usepackage{array,booktabs}
\usepackage{tabularx,tabulary}
\usepackage{longtable}
\usepackage{multirow}

% Pictures
\usepackage{graphicx}
\usepackage{wrapfig}
\usepackage{epstopdf}
\usepackage{adjustbox}

% Bibliography
\usepackage[backend=biber,
    bibencoding=utf8,
    sorting=nty,
    maxcitenames=2,
    style=numeric-comp]
{biblatex}

% Search in PDF document
\usepackage{cmap}

\usepackage{hyperref}
\usepackage[font=small,figurename=Рисунок]{caption}

% Correct quotation
\usepackage{fvextra}
\usepackage{csquotes}

% First line indent
\usepackage{indentfirst}
\usepackage{parskip}

% Symbols like copyright, degree
\usepackage{textcomp}

% Bold, italics, etc.
\usepackage{soul}

% Number of pages in document
\usepackage{lastpage}

% Local style setup
\usepackage{afterpage}

% Header and footer
\usepackage{fancyhdr}

% Linespacing
\usepackage{setspace}

% Russian letters in formulas
\usepackage{mathtext}

\usepackage{amsmath}

% Extended math symbol collection
\usepackage{amssymb}

\usepackage{amsfonts}

\usepackage{amsthm}

% Fix bugs in ams
\usepackage{mathtools}

% Math font with \mathscr command
\usepackage{mathrsfs}

% Additional math symbols
\usepackage{stmaryrd}

% Code highlighting
\usepackage[outputdir=build,newfloat]{minted}

% A comprehensive SI units package
\usepackage{siunitx}

% \nicefrac \unitfrac and \unit command
\usepackage{units}

% \qty, \norm, \abs, \div, \grad, \curl, \laplacian, etc.
\usepackage{physics}

% Graphics
\usepackage{tikz}
\usetikzlibrary{matrix}
\usepackage{pgfplots}
\pgfplotsset{compat=1.17}
\usepackage{pgfplotstable}


%=============================== Customizations ===============================
\defaultfontfeatures{Renderer=Basic,Ligatures={TeX}}
\setmainfont{CMU Serif}
\setsansfont{CMU Sans Serif}
\setmonofont{CMU Typewriter Text}

% Set linespacing
%\singlespacing
%\onehalfspacing
%\doublespacing

% Set up margins
\geometry{top=20mm}
\geometry{bottom=20mm}
\geometry{left=10mm}
\geometry{right=10mm}

% Set up headers
\pagestyle{fancy}
\renewcommand{\headrulewidth}{0mm}
\lfoot{}
\rfoot{}
\rhead{}
\chead{}
\lhead{}
%\cfoot{Нижний в центре}

\graphicspath{{images/}{build/images/}{build/}}

% Picture box margins
\setlength\fboxsep{3pt}
\setlength\fboxrule{1pt}

% First line indent
\setlength{\parindent}{0cm}
\frenchspacing

\hypersetup{
    unicode=true,
    pdftitle={Заголовок для pdf},
    pdfauthor={Автор},
    pdfsubject={Тема},
    pdfcreator={Создатель},
    pdfproducer={Производитель},
    pdfkeywords={keyword1} {keyword2},
    % False means refs in boxes, true means color refs
    colorlinks=true,
    linkcolor=red,
    citecolor=blue,
    filecolor=magenta,
    urlcolor=blue
}

\captionsetup[figure]{
    labelformat=simple,
    labelsep=endash,
    justification=centering
}

\captionsetup[listing]{
    labelformat=simple,
    labelsep=endash,
    justification=centering
}

\SetupFloatingEnvironment{listing}{name=Листинг}

% Center sections
%\allsectionsfont{\centering}

% Only show numbers if formula is referenced
%\mathtoolsset{showonlyrefs=true}

% Display formula number on the left
%\usepackage{leqno}


\addbibresource{bibliography.bib}

%=============================== Support commands ===============================
\makeatletter
\NewDocumentCommand{\swapcommands}{m m}{%
    \let\@swapcommands #1%
    \let #1 #2%
    \let #2\@swapcommands%
}
\NewDocumentCommand{\wrapensuremath}{m}{%
    % Throw error, if "\N#1" is already defined.
    \expandafter\@ifdefinable\csname #1@ensuremath\endcsname{%
        % Save old meaning
        \expandafter
        \let\csname #1@ensuremath\expandafter\endcsname
        \csname #1\endcsname
        % Define new macro
        \expandafter\edef\csname #1\endcsname{%
            \noexpand\ensuremath{%
                \expandafter\noexpand\csname #1@ensuremath\endcsname
            }%
        }%
    }%
}

% Parse curly or round brackets
\NewDocumentCommand{\@PCRB}{m m}{
    \IfNoValueTF{#1}{\IfNoValueTF{#2}{}{\quantity(#2)}}
    {#1 \IfNoValueTF{#2}{}{(#2)}}
}
% Parse math operator
\NewDocumentCommand{\parsemathoperator}{m m m}{
    \ensuremath{#1\@PCRB{#2}{#3}}
}




%=============================== Russian typography ===============================
\NewDocumentCommand{\varle}{}{\leqslant}
\NewDocumentCommand{\varge}{}{\geqslant}
\swapcommands{\epsilon}{\varepsilon}
\swapcommands{\phi}{\varphi}
\swapcommands{\kappa}{\varkappa}
\swapcommands{\emptyset}{\varnothing}
\swapcommands{\le}{\varle}
\swapcommands{\ge}{\varge}

%=============================== Ensure math for common symbols ===============================
\wrapensuremath{alpha}
\wrapensuremath{beta}
\wrapensuremath{gamma}   %\wrapensuremath{Gamma}
\wrapensuremath{delta}   %\wrapensuremath{Delta}
\wrapensuremath{epsilon}
\wrapensuremath{zeta}
\wrapensuremath{eta}
\wrapensuremath{theta}   %\wrapensuremath{Theta}
\wrapensuremath{iota}
\wrapensuremath{kappa}
\wrapensuremath{lambda}  %\wrapensuremath{Lambda}
\wrapensuremath{mu}
\wrapensuremath{nu}
\wrapensuremath{xi}      %\wrapensuremath{Xi}
\wrapensuremath{pi}      %\wrapensuremath{Pi}
\wrapensuremath{rho}
\wrapensuremath{sigma}   %\wrapensuremath{Sigma}
\wrapensuremath{tau}
\wrapensuremath{upsilon} %\wrapensuremath{Upsilon}
\wrapensuremath{phi}     %\wrapensuremath{Phi}
\wrapensuremath{chi}
\wrapensuremath{psi}     %\wrapensuremath{Psi}
\wrapensuremath{omega}   %\wrapensuremath{Omega}
\wrapensuremath{iff}
\wrapensuremath{rightarrow}
\wrapensuremath{leftarrow}
\wrapensuremath{implies}
\wrapensuremath{impliedby}

%=============================== Common math ===============================
\NewDocumentCommand{\aqty}{m}{\ensuremath{\left<#1\right>}}
\NewDocumentCommand{\n}{m}{\ensuremath{\centernot{#1}}}
\NewDocumentCommand{\x}{}{\ensuremath{\times}}
\NewDocumentCommand{\ox}{}{\ensuremath{\otimes}}
\NewDocumentCommand{\transpose}{m}{#1^\top}
\NewDocumentCommand{\T}{}{^\top}
\NewDocumentCommand{\orthogonal}{}{^\top}
\NewDocumentCommand{\ort}{}{^\top}
\NewDocumentCommand{\herm}{}{^\dagger}
\NewDocumentCommand{\suchthat}{}{\colon}
\NewDocumentCommand{\func}{m m m}{\ensuremath{#1\colon #2\rightarrow #3}} % Function: X -> Y
\NewDocumentCommand{\isomorph}{}{\ensuremath{\cong}}
\NewDocumentCommand{\divby}{}{\mathrel{\rotatebox{90}{\ensuremath{\hskip-1pt.{}.{}.}}}}
\NewDocumentCommand{\ord}{g d()}{\parsemathoperator{\operatorname{ord}}{#1}{#2}}
\NewDocumentCommand{\sgn}{g d()}{\parsemathoperator{\operatorname{sgn}}{#1}{#2}}
\NewDocumentCommand{\Id }{g d()}{\parsemathoperator{\operatorname{Id}}{#1}{#2}}

%=============================== Sets ===============================
\NewDocumentCommand{\set}{m m}{\ensuremath{\qty{#1\colon #2}}}
\NewDocumentCommand{\powerset}{g d()}{\parsemathoperator{\mathcal{P}}{#1}{#2}}
\NewDocumentCommand{\permset}{g d()}{\parsemathoperator{\mathcal{S}}{#1}{#2}}
\NewDocumentCommand{\N}{}{\ensuremath{\mathbb{N}}}
\NewDocumentCommand{\Z}{}{\ensuremath{\mathbb{Z}}}
\NewDocumentCommand{\Q}{}{\ensuremath{\mathbb{Q}}}
\NewDocumentCommand{\R}{}{\ensuremath{\mathbb{R}}}
\RenewDocumentCommand{\C}{}{\ensuremath{\mathbb{C}}} % was U+030F "COMBINING DOUBLE GRAVE ACCENT".
\NewDocumentCommand{\union}{}{\ensuremath{\cup}}
\NewDocumentCommand{\intxn}{}{\ensuremath{\cap}}
\NewDocumentCommand{\Union}{}{\ensuremath{\bigcup}}
\NewDocumentCommand{\Intxn}{}{\ensuremath{\bigcap}}

%=============================== Mathematical logic ===============================
\NewDocumentCommand{\imply}{}{\ensuremath{\:\rightarrow\:}}
\NewDocumentCommand{\implied}{}{\ensuremath{\:\leftarrow\:}}
\NewDocumentCommand{\eqdef}{}{\ensuremath{\:\longleftrightarrow\:}}
\swapcommands{\equiv}{\eqdef}
\NewDocumentCommand{\iffdef}{}{\ensuremath{\aqty{\eqdef}}} % iff by definition
\NewDocumentCommand{\eqsym}{}{\ensuremath{\eqcirc}} % Equivalent symbol by symbol
\NewDocumentCommand{\A}{m g d()}{\parsemathoperator{\forall #1\:}{#2}{#3}}
\NewDocumentCommand{\E}{m g d()}{\parsemathoperator{\exists #1\:}{#2}{#3}}
\RenewDocumentCommand{\a}{}{\ensuremath{\land}}
\RenewDocumentCommand{\o}{}{\ensuremath{\lor}}   % was little emptyset


%=============================== Group theory ===============================
\RenewDocumentCommand{\ker}{g d()}{\parsemathoperator{\operatorname{ker}}{#1}{#2}} % improvement of previous ker
\NewDocumentCommand{\imag}{g d()}{\parsemathoperator{\operatorname{im}}{#1}{#2}}
\NewDocumentCommand{\subgr}{}{\ensuremath{\leqslant}}
\NewDocumentCommand{\rsubgr}{}{\ensuremath{\geqslant}}
\NewDocumentCommand{\normsubgr}{}{\ensuremath{\trianglelefteqslant}}
\NewDocumentCommand{\rnormsubgr}{}{\ensuremath{\trianglerighteqslant}}
\NewDocumentCommand{\grcenter}{g d()}{\parsemathoperator{\operatorname{Z}}{#1}{#2}}
\NewDocumentCommand{\centralizer}{m g d()}{\parsemathoperator{\operatorname{C}_{#1}}{#2}{#3}}
\NewDocumentCommand{\normalizer}{m g d()}{\parsemathoperator{\operatorname{N}_{#1}}{#2}{#3}}
\NewDocumentCommand{\factorgr}{m m}{\ensuremath{\nicefrac{#1}{#2}}}
\NewDocumentCommand{\rfactorgr}{m m}{%
    \reflectbox{%
        \nicefrac{\reflectbox{\ensuremath{#1}}}%
        {\reflectbox{\ensuremath{#2}}}}%
}




%=============================== Theorems ===============================
\newtheoremstyle{plainbreak}
{}%                                  % Space above, empty = `usual value'
{}%                                  % Space below
{\rmfamily}%                         % Body font
{}%                                  % Indent amount (empty = no indent, \parindent = para indent)
{\bfseries}%                         % Thm head font
{}%                                  % Punctuation after thm head
{\newline}%                          % Space after thm head: \newline = linebreak
{}%                                  % Thm head spec

\theoremstyle{plainbreak}
% Name in code, name in text, counter dependency
\newtheorem{theorem}{Теорема}[section]
% Counter here is placed in between, which means that it is not dependent but the same counter
\newtheorem{proposition}[theorem]{Утверждение}
\newtheorem{corollary}{Следствие}[theorem]
\newtheorem{lemma}[theorem]{Лемма}
\newtheorem*{example}{Пример}
\newtheorem*{examples}{Примеры}
\newtheorem{problem}{Задача}[section]
\newtheorem{definition}{Определение}[section]
\newtheorem{denotation}[definition]{Обозначение}

\theoremstyle{remark}
\newtheorem*{remark}{Замечание}
\newtheorem*{solution}{Решение}


\newcommand*\@proofenvname{proof}
\newcommand*{\theoremlistshack}{%
    \leavevmode
    \ifx\@currenvir\@proofenvname
    \else
        \vspace{-\baselineskip}
    \fi
    \par%
    \everypar{\setbox\z@\lastbox\everypar{}}%
}

\makeatother


\author{Автор}
\title{Шпаргалка по \LaTeX}
\date{\today}


\begin{document}
\maketitle

\renewcommand{\abstractname}{Краткое содержание}
\begin{abstract}
    Текст аннотации
\end{abstract}

\section{Шрифт и выравнивание}
\subsection{Кегль}
% Есть 3 способа использовать эти команды
% 1. Без ничего они действуют на весь документ после себя
\tiny крошечный

\scriptsize очень маленький

\footnotesize довольно маленький

\small маленький

\normalsize нормальный

\large большой

\Large еще больше

\LARGE очень большой

% 2. Можно использовать окружение
\begin{huge}
    огромный
\end{huge}

% 3. Можно заключать в фигурные скобки вместе с командой
{\Huge громадный}
\normalsize


\subsection{Начертание}
\textbf{Жирный},
\textit{Курсив}

\emph{Выделенный текст, \emph{внутри которого выделен} другой текст}

\subsection{Выравнивание}
% Удаляем колонтитулы
\thispagestyle{empty}
% Выравнивание по центру
\begin{center}
    Федеральное государственное автономное учреждение\\
    МОСКОВСКИЙ ФИЗИКО-ТЕХНИЧЕСКИЙ ИНСТИТУТ
\end{center}
% Вертикальынй пробел
\vspace{3ex}
% Выравниванивание по правому краю
\begin{flushright}
    % Без красной строки
    \noindent
    Фамилия Имя Отчество
\end{flushright}

\begin{multicols}{2}
    Текст в двух колонках

    Текст в двух колонках

    Текст в двух колонках

    Текст в двух колонках
\end{multicols}

Пробел до конца страницы
\vfill
\begin{center}
    Москва 2014
\end{center}
\newpage

\section{Математика}
\subsection{Переходы в математический режим}

% Формула внутри текста:
Четыре --- это два плюс два: $2+2=4$

% "Выключная" формула:
Четыре --- это два плюс два: \[2+2=4\]

% Формула с номером, созданная с помощью окружения:
\begin{equation}\label{eq:def4}
    2+2=4
\end{equation}

% Пример ссылки:
\eqref{eq:def4}  на стр. \pageref{eq:def4} --- определение числа 4.

\subsection{Группирование формул}
% Для того чтобы убрать нумерацию необходимо добавить "*" к названию окружения. Выравнивание формул осуществляется по знакам равенства с помощью "&" стоящих перед ними.

% Окружение align автоматически переходит в математический режим, тогда как alined нужно окружать самостоятельно. Взамен, aligned может быть частью более сложных конструкций.
\begin{align}
    1\times 1 & = 1 & 4\times 4   & = 16 \tag{TAG} \label{eq:sq} \\
    2\times 2 & = 4 & 5\times 5   & = 25                         \\
    3\times 3 & = 9 & 60\times 60 & = 3600
\end{align}

\begin{equation}
    \begin{aligned}
        1\times 1 & = 1 & 4\times 4   & = 16   \\
        2\times 2 & = 4 & 5\times 5   & = 25   \\
        3\times 3 & = 9 & 60\times 60 & = 3600
    \end{aligned}
\end{equation}
% Пример ссылки на нестандартный тег:
В уравнении \eqref{eq:sq} на стр. \pageref{eq:sq} много слагаемых.

\subsection{Дроби: изменяющие и неизменяющие размер}


\begin{align*}
    \frac{1+\dfrac{4}{2}}{6} = 0,5
     &  &
    \frac{1+\frac{4}{2}}{6} = 0,5
\end{align*}

\subsection{Длинные формулы}
\begin{multline}
    1+2+3+4+5+6+7+\dots + \\
    + 50+51+52+53+54+55+56+57 + \dots + \\
    + 96+97+98+99+100=5050
\end{multline}

\subsection{Системы уравнений и матрицы}

\begin{align*}
    \left\{
    \begin{aligned}
        1\times x & = 1 \\
        2\times y & = 4 \\
        3\times z & = 9 \\
    \end{aligned}
    \right.
     &  &
    |x|=
    \begin{cases}
        x,  & \text{если }  x \ge 0 \\
        -x, & \text{если } x<0
    \end{cases}
\end{align*}

\begin{align*}
    \begin{pmatrix}
        a_{11} & a_{12} & a_{13} \\
        a_{21} & a_{22} & a_{23}
    \end{pmatrix}
     &  &
    \begin{vmatrix}
        a_{11} & a_{12} & a_{13} \\
        a_{21} & a_{22} & a_{23}
    \end{vmatrix}
     &  &
    \begin{bmatrix}
        a_{11} & a_{12} & a_{13} \\
        a_{21} & a_{22} & a_{23}
    \end{bmatrix}
\end{align*}

\section{Картинки}

\begin{align*}
    \includegraphics[scale=0.45]{logo}
     &  &
    \includegraphics[draft,width=\textwidth,height=3cm,keepaspectratio]{logo}
\end{align*}
\begin{figure}[h]
    \includegraphics[width=\textwidth]{eu.pdf}
\end{figure}
\begin{center}
    \adjustimage{max width=\textwidth/3, trim={{.25\width} {.25\height} {.25\width} {.25\height}},clip}{picture_source.pdf}
\end{center}
Картинки можно масштабировать и обрезать с помощью команды \verb|\adjustimage| из пакета \verb|adjustbox|.
Кроме того, не обязательно хранить картинки в бинарном формате.
Исходный \LaTeX\:код можно прекомпилировать с помощью команды:
\begin{minted}{tex}
    \immediate\write18{pdflatex -aux-directory build/logs -output-directory build/images images/picture_source.tex}
\end{minted}
Или использовать latexmk:
\begin{minted}{perl}
    push @default_files, 'images/picture_source.tex';
\end{minted}

После чего, использовать выходной файл в обычной команде импортирующей картинки.
Если в редакторе недоступны команды \verb|\immediate\write18|, то картинки нужно скомпилировать вручную по одной.
После чего положить их в папку build/images.

В десктопной версии программы GeoGebra доступен экспорт в \LaTeX\: код, использующий пакет \verb|tikz|.
Экспортируемый код использует абсолютные величины для размеров шрифта и фигур.
При этом, абсолютный размер полотна может не соответствовать реальному размеру картинки.
Поэтому, чтобы картинка скомпилировалась корректно, при экспортировании нужно с помощью подстройки параметров X Unis и Y Units подобрать желаемый размер полотна.
Этот размер полотна необходимо добавить в сгенерированный код после команды \verb|\documentclass|:
\begin{minted}{tex}
    \documentclass[10pt]{article}
    \usepackage[margin=0in,paperwidth=23cm,paperheight=15cm]{geometry}
\end{minted}
Также, GeoGebra может использовать устаревший синтаксис для отрисовки окружностей
(см. https://tex.stackexchange.com/questions/524076/problems-with-pgfplotsset).

В vim можно быстро сделать замену с помощью команды
\begin{minted}{vim}
    :%s/circle (\(\%(\d\|\.\)\+\w\+\))/circle [radius=\1]/g
\end{minted}

\section{Таблицы}

% Полноразмерные формулы в таблице задаются с помощью конструкции
% $\displaystyle ...$
% Чтобы формула не касалась границы снизу вручную задается минимальная высота строки после переноса "\\"
% Чтобы формула не касалась границы сверху изменяется параметр \extrarowheight
\setlength{\extrarowheight}{5mm}
\begin{tabular}{|cp{4cm}|} % center, right, left
    \hline
    Это предложение из многих слов. &
    Это предложение из многих слов.
    \\
    \hline
    $\displaystyle \frac{x}{y}$     &
    $\displaystyle \frac{x}{y}$
    \\[5mm] %<-задание минимальной высоты вручную
    \hline
\end{tabular}
\setlength{\extrarowheight}{0mm}
\\

% Другая команда для таблиц
% Имеет дополнительный модификатор выравнивания X
\begin{tabularx}{\textwidth}{|X|c|X|}
    \hline
    Это очень-очень длинное предложение из многих слов &
    Текст покороче                                     &
    Это очень-очень длинное предложение из многих слов Это очень-очень длинное предложение из многих слов
    \\ \hline
\end{tabularx}
\\

% Этот вариант подбирает ширину столбцов таким образом, чтобы их высота была одинаковой
% Модификаторы выравнивания пишутся с заглавной буквы. J = Justify
\begin{tabulary}{\textwidth}{|C|J|R|}
    \hline
    Это очень-очень длинное предложение из многих слов &
    Текст покороче &
    Это очень-очень длинное предложение из многих слов Это очень-очень длинное предложение из многих слов
    \\ \hline
\end{tabulary}


% Окружение \begin{table} подбирает расположение таблицы "наилучшим" образом
% В необязательном аргументе можно указать модификатор h (here), рекомендуя LaTeX расположить таблицу там же где и ссылка на нее.
% Если добавить "!", то рекомендация будет более жесткой, возможно LaTeX нарушит некоторые ограничения
\begin{table}[!h]
    \begin{center}
        \caption[Заголовок, который будет в списке таблиц]{Заголовок таблицы}\label{tab:1}
        \begin{tabular}{|c|c|c|c|}
            \hline
            0,576745371 & 0,435853468 & 0,36384912  & 0,299047979 \\
            0,064795364 & 0,028454613 & 0,751312059 & 0,693972684 \\
            \hline
        \end{tabular}
    \end{center}
    %\caption{Заголовок мог быть и здесь}
\end{table}


\begin{longtable}{|c|c|c|c|}
    \caption[Заголовок большой таблицы, который будет в списке таблиц]{Заголовок большой таблицы}
    \\ \hline
    \textbf{RND1} & \textbf{RND2} & \textbf{RND3} & \textbf{RND4} \\ \hline
    \endfirsthead
    \hline
    RND1          & RND2          & RND3          & RND4          \\ \hline
    \endhead
    \hline
    \multicolumn{4}{r}{продолжение следует\ldots} \
    \endfoot
    \hline
    \endlastfoot
    0,576745371   & 0,435853468   & 0,36384912    & 0,299047979   \\
    0,064795364   & 0,028454613   & 0,751312059   & 0,693972684   \\
    0,263563971   & 0,367508634   & 0,075536384   & 0,337780707   \\
    0,957583964   & 0,431948588   & 0,938522377   & 0,464307785   \\
    0,815740484   & 0,123129806   & 0,883432767   & 0,760983283   \\
    0,445062335   & 0,157424268   & 0,883442259   & 0,300596338   \\
    0,187159669   & 0,728663343   & 0,637199982   & 0,765684528   \\
    0,41009848    & 0,457031472   & 0,142858106   & 0,602946607   \\
    0,43315663    & 0,26058316    & 0,611667007   & 0,400328185   \\
    0,824086963   & 0,27304335    & 0,244565296   & 0,219675484   \\
    0,109578811   & 0,278478018   & 0,242519359   & 0,414669471   \\
    0,220778432   & 0,938106645   & 0,502630894   & 0,910760406   \\
    0,957583964   & 0,431948588   & 0,938522377   & 0,464307785   \\
    0,815740484   & 0,123129806   & 0,883432767   & 0,760983283   \\
    0,445062335   & 0,157424268   & 0,883442259   & 0,300596338   \\
    0,187159669   & 0,728663343   & 0,637199982   & 0,765684528   \\
    0,41009848    & 0,457031472   & 0,142858106   & 0,602946607   \\
    0,43315663    & 0,26058316    & 0,611667007   & 0,400328185   \\
    0,824086963   & 0,27304335    & 0,244565296   & 0,219675484   \\
    0,109578811   & 0,278478018   & 0,242519359   & 0,414669471   \\
    0,220778432   & 0,938106645   & 0,502630894   & 0,910760406   \\
\end{longtable}

\section{Плавающие объекты}

\begin{wrapfigure}{l}{0.3333\linewidth}
    \includegraphics[width=\linewidth]{logo}
    \caption{Картинка с обтеканием}
\end{wrapfigure}

Существует несколько эволюционных теорий, объясняющих механизмы, лежащие в основе эволюционных процессов.
В данный момент наиболее общепринятой является синтетическая теория эволюции (СТЭ), являющаяся синтезом классического дарвинизма и популяционной генетики.
СТЭ позволяет объяснить связь материала эволюции (генетические мутации) и механизма эволюции (естественный отбор).
В рамках СТЭ эволюция определяется как процесс изменения частот аллелей генов в популяциях организмов в течение времени, превышающего продолжительность жизни одного поколения.

Смотри таблицу \ref{tab:2}

\begin{wraptable}{r}{0.5\linewidth}
    \begin{tabular}{|c|c|c|c|c|c|}
        \hline
        Год  & $P_x$ & $Q_x$ & $P_y$ & $Q_y$ & $n$   \\ \hline
        2008 &       & 36    &       & 32    & —     \\ \hline
        2009 & 30    & 30    & 22    & 50    & 25 \% \\ \hline
        2010 & 36    & 30    & 22    &       & 20 \% \\ \hline
        2011 & 33    & 40    & 24    & 45    &       \\ \hline
    \end{tabular}
    \caption{Обтекаемая таблица}\label{tab:2}
\end{wraptable}

Чарлз Дарвин первым сформулировал теорию эволюции путём естественного отбора.
Эволюция путём естественного отбора — это процесс, который следует из трёх фактов о популяциях:
1) рождается больше потомства, чем может выжить;
2) у разных организмов разные черты, что приводит к различиям в выживаемости и вероятности оставить потомство;
3) эти черты — наследуемые.
Эти условия приводят к появлению внутривидовой конкуренции и избирательной элиминации наименее приспособленных к среде особей, что ведёт к увеличению в следующем поколении доли особей, черты которых способствуют выживанию и размножению в этой среде.
Естественный отбор — единственная известная причина адаптаций, но не единственная причина эволюции.
К числу неадаптивных причин относятся генетический дрейф, поток генов и мутации.

Предположение о том, что живые организмы могут изменяться, впервые встречается у греческих философов-досократиков.
Представитель милетской школы Анаксимандр считал, что все животные произошли из воды, после чего вышли на сушу.
Человек, по его представлениям, зародился в теле рыбы.
У Эмпедокла можно найти идеи гомологии и выживания наиболее приспособленных.
Демокрит считал, что наземные животные произошли от земноводных, а те в свою очередь самозародились в илу.

\section{Счетчики}

% Необязательный аргумент указывает на подчинение
\newcounter{customcounter}[section]
\setcounter{customcounter}{2}
\addtocounter{customcounter}{1}
\arabic{customcounter}
\roman{customcounter}
\Roman{customcounter}
\alph{customcounter}
\Alph{customcounter}
\Asbuk{customcounter}
\fnsymbol{customcounter}

% Переопределение нумерации разделов
\renewcommand{\thesection}{\Asbuk{section}}
\renewcommand{\thesection}{\arabic{section}}

\section{Перечни}
% Переопределение нумерации перечней
\renewcommand{\theenumi}{\roman{enumi}}

\begin{itemize}
    \item[*] Первый пункт
    \item Второй пункт
    \item
          \begin{itemize}
              \item Первый пункт вложенного списка
              \item Второй пункт вложенного списка
                    \begin{enumerate}
                        \item Первый пункт нумерованного списка
                    \end{enumerate}
          \end{itemize}
\end{itemize}

\section{Теоремы}

\begin{theorem}[Простое равенство]\label{th:1}
    $2+2=4$
\end{theorem}

\begin{proposition}
    $2+2=4$
\end{proposition}

\begin{corollary}
    $2+2=4$
\end{corollary}

\begin{problem}
$2+2$
\end{problem}

\begin{solution}
    $2+2=4$
\end{solution}

\section{Гиперссылки}

Caйт МФТИ \url{https://mipt.ru/}

\href{https://mipt.ru/}{Сайт МФТИ}

\section{Пакет babel и правила русской типографики}
\subsection{Дефисы}
N dash - Не используется в русской типографике:\\
--

M dash - Стандартное тире\\
---

Тире для отсутствующего члена предложения\\
Москва "--- столица РФ.

"--* Прямая речь

\subsection{Пробелы}

В русском наборе принято:
\begin{itemize}
    \item единицы измерения, знак процента отделять пробелами от числа: 10~кВт, 15~\%;
    \item $\tg 20^\circ$, но: 20~${}^\circ$С;
    \item знак номера, параграфа отделять от числа: №~5, \S~8;
    \item стандартные сокращения: т.\:е., и~т.\:д., и~т.\:п.;
    \item неразрывные пробелы в~предложениях.
\end{itemize}


\href{https://www.semipol.de/posts/2018/06/latex-best-practices-lessons-learned-from-writing-a-phd-thesis/}{From best practices:}

LaTeX distinguishes between different types of spaces.
Generally, a space is treated as an inter-word space as long as the character before the space is not a period.
A period is thus interpreted as the end of a sentence and the following space is assumed to be an inter-sentence space.
Thus, if you use an abbreviation with a period at the end, you have instructed LaTeX that this period does not end the current sentence by escaping the following space:
\begin{minted}{tex}
    This is a sentence w.\ a forced inter-word space.
\end{minted}
This is a sentence w.\ a forced inter-word space.

If you don’t do this, you randomly get larger spaces after abbreviations that interrupt the visual flow.
One exception to this rule is if the period is preceded by a capital letter.
In this case, LaTeX assumes an abbreviation and continues to use an inter-word space.
You can reverse this behavior by placing an \\@ before the period.
\begin{minted}{tex}
    This is done by XYZ\@. This is a new sentence.
\end{minted}
This is done by XYZ\@. This is a new sentence.

If you are using common abbreviations in English, \eg{} this one, \ie{} exempli gratia, you can use the foreign package to get commands that handle the spacing correctly:

\begin{minted}{tex}
    This is an example, \eg{} spacing is right here.
\end{minted}
This is an example, \eg{} spacing is right here.




\subsection{Пакет siunitx}
\begin{tabular}{r}
    \SI{5}{\meter\per\second}\\
    $R$ \si{\meter\per\second}\\
    R \si{\meter\per\second}\\
\end{tabular}

\section{Математические шрифты}

mathcal font: $\mathcal{ABCDEFGHIJKLMNOPQRSTUVWXYZ}$\\
mathscr font: $\mathscr{ABCDEFGHIJKLMNOPQRSTUVWXYZ}$

\section{Код на C++}
\inputminted{c++}{cpp_example.cpp}

\section{Код на \LaTeX}
\begin{minted}{tex}
    \begin{tabular}{r}
        \SI{5}{\meter\per\second}\\
        $R$ \si{\meter\per\second}\\
        R \si{\meter\per\second}\\
    \end{tabular}
\end{minted}

\section{Цитаты}
\blockquote[{\autocite{Tasnadi2006}}]{Some interesting quote.}

\blockquote[{\autocite{Spengler2012}}]{Another interesting quote.}

\blockquote[{\autocite{TheMendeleySupportTeam2011}}]{Third interesting quote.}

\subsection{Кавычки}
<<Елочки и ,,лапки``>>

\href{https://www.semipol.de/posts/2018/06/latex-best-practices-lessons-learned-from-writing-a-phd-thesis/}{From best practices:}
If you want to avoid remembering how each language and publisher style handles quotation marks, simply use the csquotes package, which automatically selects the correct way for the current document language:
\begin{minted}{tex}
    \textquote{this is the quoted text}
\end{minted}
\textquote{this is the quoted text}

If the quotation is actually from another publication that has to be cited, you can use the following shortcut to also get the correct citation through BibLaTeX:
\begin{minted}{tex}
    % 42 is the pagenumber
    \textcquote[42]{Spengler2012}{this is the quoted text}
\end{minted}
\textcquote[42]{Spengler2012}{this is the quoted text}

If your quotation is a sentence that ends with a period, some languages have different rules whether to include the period in the quotation marks or not.
csquotes can handle these cases automatically (and with configuration options), but you have to indicate the final period manually:
\begin{minted}{tex}
    \textcquote[42]{Spengler2012}[.]{This is a full sentence with period}
\end{minted}
\textcquote[42]{Spengler2012}[.]{This is a full sentence with period}

Finally, it is common to strip some parts of quoted material or add annotations to make it understandable in the current context.
csquotes provides command for these purposes so that you do not have to remember the exact rules in each language.
For instance, to replace some fraction with a summary you can use:
\begin{minted}{tex}
    \textcquote[42]{Spengler2012}[.]{This is \textelp{garbage} with  period}
\end{minted}
\textcquote[42]{Spengler2012}[.]{This is \textelp{garbage} with  period}









\section{Объявление новых команд и окружений}
% Простое объявление
\newcommand{\nw}{\ensuremath{\succcurlyeq}}

% newcommand позволяет не более 9 обязательных аргументов
% и 1 необязательный, причем необязательный аргумент всегда идет первым
% Число всех аргументов указывается в первых "[]"
% Для указания наличия необязательного агрумента добавляются вторые "[]",
% в которых указывается значение по умолчанию, возможно пустое
% Символы % для того чтобы перенос строки не считался частью команды
\newcommand{\customcommand}[2][X]{
    $#1 + #2$
}

\customcommand{Y}
\customcommand[Z]{Y}

% \renewcommand работает также, но для уже определенных команд

% Окружение отличается от команды, в основном, тем, что
% имеет завершающую последовательность
\newenvironment{customenv}[1]{
    \addtocounter{customcounter}{1}%
    Номер \thesection.\arabic{customcounter}. <<#1>> %
}%
{\vspace{0cm}} % code ending the environment

\begin{customenv}{Свое окружение}
    Текст внутри окружения.
\end{customenv}

\section{Пакет etoolbox}

\newbool{custombool}
\booltrue{custombool}
\ifbool{custombool}{Правда}{Ложь}

\listoffigures

\listoftables

\printbibliography

\end{document}
